\documentclass{article}
\usepackage{graphicx}
\usepackage{html}

\author{Bohumir~Jelinek}
\title{Yagi-logper symmetric dipole array model}

\begin{document}

\maketitle
\tableofcontents

\section{Introduction}

Yagi-logper is a program to model a Yagi or Log-periodic antennas with
with horizontal cylindrical dipoles. All the dipoles and feeding
sources need to be symmetric with respect to vertical plane. Diameter
of should be small in comparison to wavelength and other dimensions in
the antenna. These restrictions allow more efficient and precise
solution of the problem. Short theoretical description of the method
\htmladdnormallink{here}{http://www.cavs.msstate.edu/publications/350jelinek.pdf}.
Orignal Fortran routines were written by Jorgen Hald at DTU. Then I
rewrote the code in Pascal at TUKE and then in C++ at MSU as a part of
my Intro to FEA course.

\section{Input}

Input file name is provided as a command line parameter. It contains
frequency (or frequencies) in MHz, number of dipoles, geometry of
dipoles, output specifications and comments. Comments start with \#.
Each dipole is specified by length, y-coordinate, z-coordinate,
diameter, gap length, real and imaginary component of feeding voltage.

Example input file is:
\begin{verbatim}
frequency 100
dipoles 12
925 220 -330 6 0 0 0
925 -220 -330 6 0 0 0
860 0 0   6 20 10 0
725 0 72  6 0 0 0
715 0 307 6 0 0 0
705 0 552 6 0 0 0
695 0 827 6 0 0 0
685 0 1127 6 0 0 0
670 0 1457 6 0 0 0
660 0 1814 6 0 0 0
650 0 2199 6 0 0 0
640 0 2614 6 0 0 0

scaling 21 1.54 2.14

#linear gain
file amplit inimp power direct
screen amplit inimp power direct
window distrib direct
\end{verbatim}

Frequency in the above example has 21 values from 154MHz to 214MHz (it
will be made more intuitive in the next version - command
``frequencies 154 1 214'' will be introduced.)

\section{Output}

The ``file'' command (followed by the list parameters to output) in
the input file requests the output files to be created. Possible
outputs are ``amplit'' (current amplitudes), ``inimp'' (input
impedance), ``power'' (power radiated by dipoles) and ``direct''
(directivity). Directivity (and current distributions in the near
future) can be displayed in the graphics window, as shown on Figures
\ref{fig:e48} and \ref{fig:h48}.

\begin{figure}
  \includegraphics[width=0.95\textwidth]{e48}
  \caption{\label{fig:e48}E-plane directivity pattern of 15 element
    Yagi antenna.}
\end{figure}

\begin{figure}
  \includegraphics[width=0.95\textwidth]{h48}
  \caption{\label{fig:h48}H-plane directivity pattern of 15 element
    Yagi antenna.}
\end{figure}

\end{document}
